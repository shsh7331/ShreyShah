\documentclass[sigconf,twocolumn,11pt]{acmart}
\AtBeginDocument{\fontsize{11}{13.2}\selectfont} % 11pt font with 1.1 spacing

\title{Analyzing the Correlation Between Crime and Temperature in Chicago}

\author{Patrick Ridley}
\email{pari8094@colorado.edu}
\affiliation{
  \institution{University of Colorado Boulder}
  \country{USA}
}

\author{Shrey Shah}
\email{shsh7331@colorado.edu}
\affiliation{
  \institution{University of Colorado Boulder}
  \country{USA}
}
\begin{abstract}
Crime patterns in urban areas are influenced by various environmental factors, with temperature fluctuations playing a significant role in the frequency of violent crimes. Previous research has shown a correlation between higher temperatures and increased instances of violent crimes, such as assaults and homicides. This study aims to analyze the relationship between temperature variations and violent crime occurrences in Chicago from 2010 to 2025. By leveraging publicly available crime and weather datasets, we seek to uncover patterns that can provide insight into how temperature impacts criminal activity. Through data mining techniques, we will explore whether specific weather conditions contribute to crime surges, offering a data-driven perspective on crime prevention.


To achieve this, we will employ various data analysis methods, including exploratory data analysis (EDA), clustering, and association rule mining. Data preprocessing will ensure consistency between crime and weather datasets, aligning timestamps and handling missing values. Clustering techniques will be applied to detect crime trends and seasonal variations, while correlation analysis will measure the strength of the relationship between temperature and crime. Additionally, association rule mining will identify recurring patterns in violent crime occurrences under different temperature conditions.


The findings from this study will provide valuable insights for law enforcement, urban planners, and policymakers in designing strategies for crime prevention and resource allocation. By identifying when and where violent crimes are more likely to occur based on temperature patterns, law enforcement agencies can optimize patrol scheduling and emergency response planning. Furthermore, city officials can use this information to implement community safety initiatives and temperature-based crime intervention strategies. This research contributes to a deeper understanding of environmental influences on crime and highlights the importance of integrating climate factors into public safety planning.
\end{abstract}
\begin{document}

\maketitle

\section{Problem Statement and Motivation}
Crime patterns in urban areas are influenced by a range of environmental and socioeconomic factors, with temperature fluctuations emerging as a potential driver of violent crime rates. Several studies suggest that warmer temperatures correlate with increased occurrences of violent crimes, such as assaults, robberies, and homicides. This phenomenon is often attributed to increased outdoor activity, heightened aggression due to heat, and greater social interactions that can lead to conflicts. In contrast, colder temperatures may deter outdoor interactions, reducing the likelihood of violent encounters.

This project aims to analyze historical crime and weather data in Chicago from 2010 to 2025 to uncover patterns between temperature variations and violent crime rates. By applying data mining techniques, we seek to determine whether specific temperature conditions influence crime surges or reductions. Through clustering, association rule mining, and time-series analysis, we will examine seasonal trends, geographic hotspots, and recurring patterns in violent crime occurrences.

The goal is to provide data-driven insights that can assist law enforcement, urban planners, and policymakers in crime prevention and resource allocation strategies. By identifying when and where violent crimes are most likely to occur under different temperature conditions, this research can help optimize police staffing, emergency response preparedness, and community intervention programs to enhance public safety.

\section{Literature Survey}
Several studies have examined the impact of weather on crime trends:
\begin{itemize}
    \item \textbf{Anderson (1987) - The Heat Hypothesis}: Suggests that violent crime increases in higher temperatures due to heightened aggression. Analyzed multiple cities, including Chicago, to validate these findings.
    \item \textbf{The Association Between Weather and Daily Shootings in Chicago (2012--2016)}: Found that significant increases in temperature are associated with higher shootings, breaking down trends by weekdays, weekends, and holidays.
    \item \textbf{Harries et al. (1984) - Property Crime and Weather Relationship}: Found that property crimes have mixed relationships with temperature, with some increasing during cold weather. Conducted in Vancouver and Ottawa, showing that different climate zones experience varying crime-weather correlations.
    \item \textbf{The Urban Crime and Heat Gradient in High and Low-Poverty Areas}: Reviews temperature shifts and poverty levels in crime rates. While not the primary focus of our study, it suggests an additional socioeconomic layer we may explore.
    \item Most prior studies rely on statistical regression models without incorporating modern data mining techniques such as clustering, association rule mining, and predictive modeling. Our approach differentiates itself by applying these techniques to large-scale data while integrating crime type, location, and severity alongside temperature variations.
\end{itemize}

\section{Proposed Work}
\subsection{Data Preprocessing and Cleaning}
\begin{itemize}
    \item Remove duplicate and irrelevant records.
    \item Filter out nonviolent crimes (e.g., white-collar crimes, shoplifting).
    \item Exclude cases where an arrest was not performed.
    \item Handle missing values in crime and weather datasets.
    \item Normalize date/time formats for merging.
\end{itemize}

\subsection{Data Attributes}
\begin{itemize}
    \item \textbf{Nominal:} Crime type, weather condition.
    \item \textbf{Numerical}: Temperature, crime count, crime severity index.
    \item \textbf{Boolean}: Arrest made (yes/no).
    \item \textbf{Time-Based}: Date, season, time of crime.
    \item \textbf{Derived Attributes}: Processed data used for trend analysis.

\end{itemize}


\subsection{Data Integration and Handling Time Granularity Issues}
\begin{itemize}
    \item Align timestamps by aggregating crime counts to daily totals.
    \item Aggregate weather data into daily min/max/average temperature.
    \item Merge crime and temperature datasets based on date and crime type.
\end{itemize}

\subsection{Database Indexing and Optimization}
\begin{itemize}
    \item Primary Indexing: Based on date and location (latitude/longitude) for optimized lookups.
    \item Secondary Indexing: On crime type and severity to speed up queries.
    \item Partitioning Strategy: Monthly or yearly partitions for efficiency.
\end{itemize}

\section{Data Mining Techniques}
\subsection{Descriptive Analytics: Exploratory Data Analysis (EDA)}
\subsubsection{Visualization}
\begin{itemize}
    \item Scatter plots: Temperature high/low vs. crime rate.
    \item Trend lines: Crime rate over time and temperature variation.
\end{itemize}
\subsubsection{Data Cube Analysis:}
\begin{itemize}
    \item Violent crimes vs. temperature highs/lows.
    \item Crimes by month/season.

\end{itemize}

\subsection{Clustering: Crime Pattern Detection}
\begin{itemize}
    \item Clustering techniques will help identify hidden crime patterns that are not apparent in raw data.
\end{itemize}
\subsubsection{K-Means Clustering}
\begin{itemize}
    \item Groups crime incidents based on temperature conditions.
    \item Identifying distinct patterns at low, moderate, and high temperatures.
    \item Helps detect seasonal crime trends (e.g., more assaults in summer).

\end{itemize}

\subsection{Association Rule Mining}
\begin{itemize}
    \item Frequent pattern mining techniques will identify strong relationships between crime occurrences and weather patterns.
\end{itemize}
\subsubsection{Apriori Algorithm}
\begin{itemize}
    \item Extracts rules such as “Higher temperatures increase violent crime frequency.”
\end{itemize}
\subsubsection{FP-Growth Algorithm}
\begin{itemize}
    \item Analyzes frequent crime patterns across temperature highs, lows, and seasonal shifts.
\end{itemize}

\subsection{Time-Series Analysis: Forecasting Crime Trends}
\begin{itemize}
    \item Time-series techniques will be used to analyze seasonal crime fluctuations over time.
\end{itemize}
\subsubsection{ARIMA (AutoRegressive Integrated Moving Average)}
\begin{itemize}
    \item Models and forecasts crime rate trends based on historical data.
\end{itemize}
\subsubsection{Seasonality Decomposition}
\begin{itemize}
    \item Identifies recurring crime fluctuations across different seasons.
\end{itemize}

\section{Datasets}
\subsubsection{\textbf{Chicago Crime Data}}
\begin{itemize}
    \item City of Chicago Open Data, 1M+ records from 2010--present.
    \item\href{https://data.cityofchicago.org/Public-Safety/Crimes-2001-to-Present/ijzp-q8t2/about_data}{Chicago Crime Data}
\end{itemize}
\subsubsection{\textbf{Chicago Weather Data}}
\begin{itemize}
    \item NOAA National Centers for Environmental Information.Daily temperature and climate data in high/low format.
    \item\href{https://www.ncdc.noaa.gov/cdo-web/datasets/LCD/stations/WBAN:14819/detai}{Chicago Weather Data}
\end{itemize}

\section{Evaluation Methods}
\begin{itemize}
    \item Statistical Correlation Metrics: Pearson Correlation, Spearman’s Rank Correlation, P-Value Significance Testing.
    \item Clustering Evaluation: Silhouette Score, Davies–Bouldin Index.
    \item Time-Series Analysis Validation: Baseline Trend Comparison, Rolling Window Analysis, Seasonality Strength Measurement.
\end{itemize}

\section{Tools and Technologies}
\subsubsection{\textbf{Programming}}
\begin{itemize}
    \item Python (Pandas, NumPy, Matplotlib).
\end{itemize}

\subsubsection{\textbf{ Database}}
\begin{itemize}
    \item MySQL for structured relational data.
\end{itemize}

\subsubsection{\textbf{Visualization}}
\begin{itemize}
    \item Matplotlib for trend visualizations in Jupyter Notebook.
\end{itemize}

\section{Milestones and Timeline}
\begin{table}[H]
    \centering
    \begin{tabular}{|c|l|}
        \hline
        \textbf{Date} & \textbf{Task} \\
        \hline
        Feb 3 & Project Proposal \\
        Feb 17 & Peer Feedback \\
        Mar 3 & Proposal Paper \\
        Mar 17 & Progress Report \\
        Apr 28 & Code \& Descriptions \\
        Apr 28 & Presentation \\
        Apr 28 & Peer Evaluation \& Interview \\
        \hline
    \end{tabular}
\end{table}

\section{Acknowledgements}
\begin{itemize}
    \item We also extend our appreciation to the City of Chicago and the National Centers for Environmental Information (NOAA) for providing publicly available crime and weather datasets. These datasets form the foundation of our research, enabling us to explore the correlation between temperature and violent crime trends.
    
\end{itemize}

\begin{thebibliography}{99}


\bibitem{anderson1987}
C. A. Anderson. "Temperature and aggression: Ubiquitous effects of heat on occurrence of human violence." 
\textit{Psychological Bulletin}, vol. 102, no. 1, pp. 27–51, 1987.


\bibitem{harries1984}
K. D. Harries, S. J. Stadler, and R. T. Zdorkowski. "Seasonality and Assault: Explorations in Inter-neighborhood Variation, Dallas 1980." 
\textit{Annals of the Association of American Geographers}, vol. 74, no. 4, pp. 590-604, 1984.


\bibitem{ranson2014}
M. Ranson. "Crime, weather, and climate change." 
\textit{Journal of Environmental Economics and Management}, vol. 67, no. 3, pp. 274-302, 2014.


\bibitem{chicago_crime_data}
City of Chicago. "Chicago Crime Data (2001–Present)." 
Available: \url{https://data.cityofchicago.org/Public-Safety/Crimes-2001-to-Present/ijzp-q8t2/about_data}. Accessed: Feb. 2025.


\bibitem{noaa_weather}
National Centers for Environmental Information (NOAA). "Chicago Weather Data." 
Available: \url{https://www.ncdc.noaa.gov/cdo-web/datasets/LCD/stations/WBAN:14819/detail}. Accessed: Feb. 2025.

\bibitem{han2011}
J. Han, M. Kamber, and J. Pei, \textit{Data Mining: Concepts and Techniques}. 3rd ed., Morgan Kaufmann, 2011.

\end{thebibliography}

\end{document}
